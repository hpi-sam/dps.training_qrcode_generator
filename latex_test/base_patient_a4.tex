\documentclass[a4paper,oneside]{article}
\usepackage[utf8]{inputenc}
\usepackage[forget]{qrcode}
\usepackage{tikz}
\usetikzlibrary{positioning}
\usepackage{graphicx}
\usepackage[left=2cm,top=2cm,right=2cm,bottom=2cm,verbose,nohead,nofoot]{geometry}

\begin{document}

\pagestyle{empty}
\begin{center}
	\begin{tikzpicture}
		% Background Image 
		\node [opacity=0.5] (Patient) {%
			\includegraphics[height=0.88\textheight]{image_patient}
		};
		
		% Title
		\node [above=of Patient,font=\huge,scale=2] () {\sffamily Patient \textbf{\Huge A} 20 P - 1 D XA};
		
		% QR Code
		\node [scale=2.2] at (0,2.5) (QR) {\qrcode[level=M]{container-1-1}};

		% Note "Gehfähig"
		\node [below=3cm of QR, anchor=north, rounded corners=1cm, line width=0.2cm, draw=purple, color=purple, fill=purple, fill opacity=0.2, text opacity=1, draw opacity=1, minimum width=1.5cm, minimum height=1.5cm,font=\Huge, scale=4] (G) {\sffamily \textbf{G}};
		\node [above=0.5cm of G.south, anchor=south, color=purple, font=\Huge] () {\sffamily Gehfähig};
		
		% Footer / Label
		\node [minimum width=\textwidth, below=-1cm of Patient] (BaseLabel) {};
		\node [anchor=south east, left=0cm of BaseLabel.east, font=\tiny, text centered, text width=3cm] (Label) {
		 	{\sffamily \large dps.training} \\ 
			Software \textcopyright{}HPI \\ 
			Datensatz \copyright{}BBK/BABZ \\
			manv-simulation.de
		};
		\node [left=0cm of Label, anchor=east] (ImgLabel) {\includegraphics[width=1.2cm, trim=7cm 7cm 7cm 7cm,clip]{icon_player}};
		
	\end{tikzpicture}%
\end{center}%



\end{document}

